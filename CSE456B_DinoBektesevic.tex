\documentclass{article}
\usepackage{amsmath,amsfonts,amsthm,amssymb,amsopn,bm}
\usepackage[margin=.9in]{geometry}
\usepackage{graphicx}
\usepackage{url}
\usepackage[usenames,dvipsnames]{color}
\usepackage{fancyhdr}
\usepackage{multirow}
\usepackage{listings}
\usepackage{hyperref}

\definecolor{keywords}{RGB}{255,0,90}
\definecolor{comments}{RGB}{0,0,113}
\definecolor{red}{RGB}{160,0,0}
\definecolor{green}{RGB}{0,150,0}
 
\lstset{language=Python, 
        basicstyle=\ttfamily\tiny, 
        keywordstyle=\color{keywords},
        commentstyle=\color{comments},
        stringstyle=\color{red},
        showstringspaces=false}

\newcommand{\field}[1]{\mathbb{#1}}
\newcommand{\1}{\mathbf{1}}
\newcommand{\E}{\mathbb{E}} 
\renewcommand{\P}{\mathbb{P}}
\newcommand{\R}{\field{R}} % real domain
% \newcommand{\C}{\field{C}} % complex domain
\newcommand{\F}{\field{F}} % functional domain
\newcommand{\T}{^{\textrm T}} % transpose
\def\diag{\text{diag}}

%% operator in linear algebra, functional analysis
\newcommand{\inner}[2]{#1\cdot #2}
\newcommand{\norm}[1]{\left\|#1\right\|}
\newcommand{\twonorm}[1]{\|#1\|_2^2}
% operator in functios, maps such as M: domain1 --> domain 2
\newcommand{\Map}[1]{\mathcal{#1}}
\renewcommand{\theenumi}{\alph{enumi}} 

\newcommand{\Perp}{\perp \! \! \! \perp}

\newcommand\independent{\protect\mathpalette{\protect\independenT}{\perp}}
\def\independenT#1#2{\mathrel{\rlap{$#1#2$}\mkern2mu{#1#2}}}
\newcommand{\vct}[1]{\boldsymbol{#1}} % vector
\newcommand{\mat}[1]{\boldsymbol{#1}} % matrix
\newcommand{\cst}[1]{\mathsf{#1}} % constant
\newcommand{\ProbOpr}[1]{\mathbb{#1}}
\newcommand{\points}[1]{\small\textcolor{magenta}{\emph{[#1 points]}} \normalsize}
\date{{}}

\setlength\parindent{0px}

\begin{document}
\title{Homework \#0 B}
\author{\normalsize{Spring 2020, CSE 446/546: Machine Learning}\\
\normalsize{Dino Bektesevic}}
\maketitle

Collaborated: Conor Sayers, Joachim Moeyenes, Jessica Birky, Leah Fulmer

\section*{Probability and Statistics}
B.1  \points{1} Let $X_1,\dots,X_n$ be $n$ independent and identically distributed random variables drawn uniformly at random from $[0,1]$. If $Y = \max\{X_1,\dots,X_n\}$ then find $\E[Y]$.\\

\section*{Linear Algebra and Vector Calculus}
B.2 \points{1} The \textit{trace} of a matrix is the sum of the diagonal entries; $Tr(A) = \sum_i A_{ii}$. If $A\in\mathbb{R}^{n\times m}$ and $B\in\mathbb{R}^{m\times n}$, show that $Tr(AB) = Tr(BA)$.\\
It can be shown in general that trace is invariant under cyclic permutations, which for case of n=2 looks like commutation:
$$\text{tr}(AB)=\sum_i^m (ab)_{ii} = \sum_{i=1}^m\sum_{j=1}^n a_{ij}b_{ji}= \sum_{j=1}^n\sum_{i=1}^m b_{ji}a_{ij} = \sum_i^n (ab)_{jj} = \text{tr}(BA)$$
using the fact that product of $n\times m$ and $m\times n$ matrix is an $m\times m$ matrix and that the trace of a square matrix can be rewritten as:
$$\text{tr}(A^TB)=\sum_{i,j}A_{ij}B_{ij}$$
as per Wikipedia.\\

B.3 \points{1} Let $v_1,\dots,v_n$ be a set of non-zero vectors in $\mathbb{R}^d$. Let $V = [v_1,\dots,v_n]$ be the vectors concatenated. 
    \begin{enumerate}
        \item What is the minimum and maximum rank of $\sum_{i=1}^n v_i v_i^T$? \\
        Vector $v$ is a column vector with n rows. Transpose of vector $v$ will have n columns and 1 row. The matrix resulting from matrix multiplication has the number of rows of the first and the number of columns of the second matrix, i.e. it will be an $(n, n)$ square matrix. Basis of an $\mathbb{R}^d$ is spanned by $d$ linearly independent vectors. So the maximum rank of a matrix could be d when $n\geq d$. Otherwise the maximal rank of the sum is $n$ when $0<n<d$. If all vectors are the same unit vectors then the minimum rank could be 1. 
        
        \item What is the minimum and maximum rank of $V$? \\
        Same as above with the additional comment that minimal rank of a matrix would be 0 in case of a zero matrix, but since the vectors $v$ must be non-zero the minimal rank of V would again be 1 in the case all $v$ are unit vectors, or if all given vectors are linearly dependent.
        
        \item Let $A \in \mathbb{R}^{D \times d}$ for $D > d$. What is the minimum and maximum rank of $\sum_{i=1}^n (A v_i) (A v_i)^T$? \\
        Rank can not be greater than either number of rows or columns, i.e. $\text{Rank}(A)\leq \min(i,j)$. So the rank will be limited to $d$ despite $D>d$. Rank will be preserved is A is monomorphism, i.e. if the operator represented by $A$ is injective. Assuming an monomorphic map we can repeat everything stated above - the rank of V is set between $1$ and $d$ depending on the choice of $v_i$ vectors. In any other situation the rank of A can only be less than $d$ as it must project different values from its domain into the same value in its codomain. 
        
        \item What is the minimum and maximum rank of $AV$? What if $V$ is rank $d$? \\
        The number of columns in the first matrix must be equal to the number of rows in the second matrix, so the product of $AV$ results in an $(D,d)x(d, n)=(D,n)$ matrix. Following what was said above the rank will be limited to the smaller of the two values $D$ or $n$. The maximum rank of the product $AV$ will be limited by the choice of $v$'s and the properties of mapping $A$ represents. 
    \end{enumerate}
\end{document}
