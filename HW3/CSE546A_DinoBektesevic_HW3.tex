\documentclass{article}
\usepackage{amsmath,amsfonts,amsthm,amssymb,amsopn,bm}
\usepackage[margin=.9in]{geometry}
\usepackage{graphicx}
\usepackage{url}
\usepackage[usenames,dvipsnames]{color}
\usepackage{fancyhdr}
\usepackage{multirow}
\usepackage{listings}
\usepackage{hyperref}

\definecolor{keywords}{RGB}{255,0,90}
\definecolor{comments}{RGB}{0,0,113}
\definecolor{red}{RGB}{160,0,0}
\definecolor{green}{RGB}{0,150,0}
 
\lstset{language=Python, 
        basicstyle=\ttfamily\tiny, 
        keywordstyle=\color{keywords},
        commentstyle=\color{comments},
        stringstyle=\color{red},
        showstringspaces=false}

\newcommand{\argmax}{\arg\!\max}
\newcommand{\argmin}{\arg\!\min}
\newcommand{\field}[1]{\mathbb{#1}}
\newcommand{\1}{\mathbf{1}}
\newcommand{\E}{\mathbb{E}} 
\renewcommand{\P}{\mathbb{P}}
\newcommand{\N}{\mathcal{N}} % NormalDist
\newcommand{\R}{\field{R}} % real domain
% \newcommand{\C}{\field{C}} % complex domain
\newcommand{\F}{\field{F}} % functional domain
\newcommand{\T}{^{\textrm T}} % transpose
\def\diag{\text{diag}}

%% operator in linear algebra, functional analysis
\newcommand{\inner}[2]{#1\cdot #2}
\newcommand{\norm}[1]{\left\|#1\right\|}
\newcommand{\twonorm}[1]{\|#1\|_2^2}
% operator in functions, maps such as M: domain1 --> domain 2
\newcommand{\Map}[1]{\mathcal{#1}}
\renewcommand{\theenumi}{\alph{enumi}} 

\newcommand{\Perp}{\perp \! \! \! \perp}

\newcommand\independent{\protect\mathpalette{\protect\independenT}{\perp}}
\def\independenT#1#2{\mathrel{\rlap{$#1#2$}\mkern2mu{#1#2}}}
\newcommand{\vct}[1]{\boldsymbol{#1}} % vector
\newcommand{\mat}[1]{\boldsymbol{#1}} % matrix
\newcommand{\cst}[1]{\mathsf{#1}} % constant
\newcommand{\ProbOpr}[1]{\mathbb{#1}}
\newcommand{\points}[1]{\small\textcolor{magenta}{\emph{[#1 points]}} \normalsize}
\date{{}}

\setlength\parindent{0px}

\begin{document}
\title{Homework \#3 A}
\author{\normalsize{Spring 2020, CSE 446/546: Machine Learning}\\
\normalsize{Dino Bektesevic}}
\maketitle

Collaborated: Conor Sayers, Joachim Moeyenes, Jessica Birky, Leah Fulmer

\section*{Conceptual Questions \points{10} }
\begin{enumerate}
    \item \points{2} alalala
\end{enumerate}



\section*{Convexity and Norms \points{10}}

\begin{align*}
\centering
    -|a|\leq a &\leq |a| \\
    -|b|\leq b &\leq |b|
\end{align*}

\vskip 2cm

%\lstinputlisting[linerange={0-269,452-454}]{HW2_code_solutions/coordinate_descent.py}  


%\begin{table}[h!]
%\centering
    %\begin{tabular}{l|c}
     %\centering
 %       Feature Name              &  Magnitude \\
 %       \hline
 %       PctKids2Par               & -0.07017508 \\
 %       PctHousOccup              & -0.00724790 \\
 %       PctWorkMom                & -0.00544586 \\
 %       agePct12t29               & -0.00367240 \\
 %       LemasPctOfficDrugUn       & 0.00051549 \\
 %       PctVacantBoarded          & 0.00126885 \\
 %       pctUrban                  & 0.01042646 \\
 %       MalePctDivorce            & 0.01102267 \\
 %       NumStreet                 & 0.01575945 \\
 %       HousVacant                & 0.02049849 \\
 %       PctPersDenseHous          & 0.03062536 \\
 %       PctIlleg                  & 0.06802481 
 %   \end{tabular}{}
%\end{table}{}


\end{document}
